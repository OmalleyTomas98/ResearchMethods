\documentclass[journal]{IEEEtran}
\usepackage[utf8]{inputenc}

\usepackage[sorting=none]{biblatex}
\addbibresource{bibliography.bib}

\begin{document}

\markboth{Ounantum Computing : The Next Step in Computing~2020}%
{5G: The Next Step in Mobile Communications, November~2017}

\title{Quantum computing: The Next Step in Computing}
\author{Tomás O'Malley ,~\IEEEmembership{Software Development (Honours),~GMIT}%
}
% Author: Tomás O'Malley 3611288@gmail.com
% Student of Galway-Mayo Institute of Technology, Department of Computer Science and Applied Physics
% Literature Review on Quantum Computing  completed as part of Research Methods in Computing and IT.

\maketitle

\begin{abstract}
Qunantum Computing

\end{abstract}
\begin{IEEEkeywords}
Qunantum Computing, classical computing, ready, mechanics.
\end{IEEEkeywords}

\section{Introduction}


\subsection{A Brief History of Computing }


\subsection{Next  Generation and Current Limitations} \label{subsec:4g}


\section{Next Generation}


\subsection{The Definition of 	Quantum computing} \label{subsec:def5g}

%\textsc{1) Calculations: }
\subsubsection{Calculations}


%\textsc{2) Medical Fields: }
\subsubsection{Medical Fields}


%\textsc{3) Research and Development : }
\subsubsection{Research and Development }


\section{Proposed Technologies}


\subsection{Transmons}


\subsection{ Logic gates}\label{subsec: logicgates}


\subsection{Qubits}


\subsection{energy}\label{subsec:energy}



\section{The Future of Quantum Computing }


\subsection{Problems Facing Quantum Computing }


\subsubsection{Technological limatations  }
%\textsc{1) Technological limatation: }


%\textsc{2) Infrastructure: }
\subsubsection{Infrastructure}


\subsection{Possible Use Cases}


\subsubsection{In Day-to-Day  Life}


\subsubsection{The Bigger Picture}


\section{Conclusion}


\bigskip
\bigskip

\printbibliography
\end{document}